\documentclass[11pt, twocolumn]{article}

% \usepackage{amsmath}
% \usepackage{hyperref}
% \usepackage{graphicx}
% \usepackage{subfigure}
% \usepackage{indentfirst} % indent frst paragraph of section
%\usepackage{tocbibind} %adds list of figures painlesly to ToC
% \usepackage{multirow} %in tables
% \usepackage{caption} % in tables
% \usepackage[usenames,dvipsnames]{color}
\usepackage[scale=.75]{geometry} % top=1in, bottom=1in, left=1in, right=1in
\usepackage{lipsum}

\newcommand{\ts}{\textsuperscript}
\newcommand{\HRule}{\rule{\linewidth}{0.5mm}}


\begin{document}
% \vspace*{\fill}

\twocolumn[\begin{@twocolumnfalse}
\begin{center}
    \begin{large}
    {\HRule \\[0.2cm]}
    \textsc{\Huge SMARTer 2020 review}
    {\HRule \\[0.3cm]}
    \end{large}

    \begin{minipage}{ 0.49\textwidth }
        \begin{flushleft}
            Kacper \textbf{Sokol}---\texttt{ks1591}---4GGK1\\
            \today
        \end{flushleft}
    \end{minipage}
    \begin{minipage}{ 0.49\textwidth }
        \begin{flushright}
            {Sustainability, Technology and Business\\
             COMSM0006; University of Bristol, UK\\[0.3cm]}
        \end{flushright}
    \end{minipage}
\end{center}
\vspace*{1em}\end{@twocolumnfalse}]

% \vspace*{3cm}

% \begin{center}
  % \line(1,0){250}
% \end{center}

% \begin{abstract}
  % \lipsum[5]
  % \begin{center}
    % Keywords: \textbf{A, B, C}
  % \end{center}
% \end{abstract}
% \vspace*{\fill}

% \thispagestyle{empty}
% \newpage
% \tableofcontents

% \thispagestyle{empty}
% \newpage
% \setcounter{page}{1}


Uncertainty Analysis
In particular we would like you to evaluate epistemic uncertainty that arises from assumptions applied to the modelling regarding\\
- quality of the data used\\
- model structure\\
- obstacles and enabling factors that might affect the uptake of innovations (financial, technical, social)\\
- implicit assumptions in system boundaries and methodology (for example rebound effect) In order to reduce this uncertainty we’d like you to:\\
- reconstruct the estimates of the abatement potential in a tool\\
- provide a critical analysis of the Smarter 2020 estimates in terms of the four types of uncertainty. This analysis should include identifying the data sources used in the SMARTer 2020 report.

\line(1,0){100}

Master Level Analysis
If your team includes master level students we additionally would like to receive improved estimates for the abatement potential of at least 5 sublevers (or the creation of new sublevers) through
- better data sources
- a more detailed modelling structure\\

Masters level students should also implement their improved abatement potential estimates in the tool.\\

Typically 12 A4 pages, + tool

\section{A}
For each intervention, calculate the 'addressable emissions'. (eg total fuel expended on transportation).
Estimate a savings potential for the technology based on;   Efficiency improvement estimate for the technology. stimate of the rollout of that technology by 2020.
These figures found from the literature, and estimates by experts.

\subsection{eCommerce}
Total emissions from private transport estimated by the IEA to be 5.7GtCO2e in 2020 ->\\
Estimate that 20\% of this is for shopping trips. (Found in Wall Street Journal blog on eCommerce.) ->\\
Assume that 15\% of shopping is replaced by eCommerce in developed countries, and 0\% in developing. ->\\
This gives 7.5\% actual abatement of shopping trips = 0.09 -> \\
Ignore transportation emissions associated with eCommerce delivery.\\
\subsection{Motor speed optimisation}
Total emissions from electrical motors estimated to be 2.92 GtCO2e. ->\\
Expert interviews for SMART2020 estimate that: variable speed control could increase efficiency by 30\%. \& this technology will penetrate 60\% of the market by 2020. ->\\
Hence total reduction potential of 0.53 GtCO2e
\subsection{Integration of Renewables}
IEA scenario estimates that renewables could be responsible for 3.4 GtCO2e savings of emissions by 2020. ->\\
Estimate (how? Unclear.) that 25\% credit for this is given to the role of ICT in integrating this technology into the power mix.

\section{B}

Analysis ignores direct impact of ICT for each solution (though the report does calculate the overall direct impact of ICT.)\\

Analysis (mostly) ignores any rebound effects.\\

This approach 'scoping' --- would not be anywhere near robust enough to assess and monetize the impacts. Very simplistic.\\

IF ICT-enabled projects are to be truly assessed, then clear methodologies are needed for doing this. GeSI and the Carbon Trust are working on such methodologies.\\
















\newpage
\section{Introduction}
\begin{itemize}
\item global warming is society's main problem --- main contribution is GHG --- $CO_2$ is main part of GHG --- caused by burning fossil fuels ---

\item decouple emission growth from economic growth

\item introducing this may lead to additional jobs and money savings

\item they claim total of $\mathbf{9.1} \mathbf{Gt}CO_2\mathbf{e}$ but it only sums up to $\mathbf{9} \mathbf{Gt}CO_2\mathbf{e}$ on the other hand figure shown \emph{Manufacturing: 1.3} but the text only 1.2. And again page 13 sais 1.2. WTF?

\item after introducing ICT we do not take into account its influence on given sector; event thou it introduces additional source of GHG it gives an advantage over potential savings in GHG.

\item the estimates are inacurate as new technologies are created constantly giving rise to new abatement posibillities

\item focus on \emph{Power} end-use sector as it has lion's share
\end{itemize}

{\small
Abbreviations:\\
GHG --- Green House Gases\\
ICT --- Information Communications Tecnology\\
$CO_2e$ --- $CO_2$ equivalents\\
BAU --- business as usual
}

\section{Levers reconstruction}
\begin{itemize}
\item focus on \emph{Power} end-use sector as it has lion's share

\item can be divided into: \emph{integration of renewables}, and \emph{smart grid enabling}
\item first: Demand management, Time-of-day pricing, Power-load balancing, Power grid optimization
\item second: Integration of renewables in power generation, Virtual power plant, Integration of off-grid renewables and storage

\item In china alone on whole energy sector $390Mt CO_2e$ can be avoided

\item in China government intervention is needed; to low energy prices so people dont wnat to do anything; grid infrastructure update needed to facilitate rapid changes to ICT enabled solutions.
\item thay claim that [power grid optimization] has the greatest potential but do not describe it in any details
\item ``Business case assasment'' --- not clear how they arrive at circle 'filling' and what is the 'equation' to produce the overall score --- no clear weights

\item in China scale of the vertical axis on 'Attractiveness of business model---Fig 42' does not agree with previous estimates

\item lack of strong business cse for majority fo sublevers --- chence government push needed

\item due to government monopoly --- lack of competition --- --- its has been an issue for a long time and people were aware of it nd noothing happened

\item additional peak power plant to support demand, which otherwise stay idle
\item government try to secure peak supply rather than shift demand


\item U.S.A.
\item high energy use, dirty energy mix, old inefficient grid
\item energy is 42\% of all US emissions
\item abatement potiential is $350 MtCO_2e$
\item dirty power mix --- coal is 37\% of total energy generation --- generally $0.67 kgCO_2e / kWh$
\item emision from energy generation decreases as they use less coal and more clean sources like natural gas --- from $4.3 GtCO_2e$ in 2000 to $4.3 GtCO_2e$ in 2010
\item the grid has not been improved so is inefficient
\item domestic oil-and-gas resources as well as low energy taxes cause low energy prices, hence low interest in renevable energies
\item there are little general law driving renewables like ITC and PTC; majority of decision is up to particlar state; howewer net metering policy is used so that you can sell the overproduced energy for retail price to the grid

\item Main obstacle is weak business case for maority of Energy sublevers in USA. Hence government policy are needed and subsidiaries to boost the market.
\item \emph{Integration of renewables in power generation} and \emph{power grid optimization} together yield 82\% of abatement potential chence stress should be but on it
\item not clear how \emph{overall business case} is estimated --- neither majority vote
\item \emph{economics}, \emph{full deployment}, and \emph{aligned incentives} barriers.
\item to introduce they are to high in cost and dont have enough return
\item investment in research to reduce costs---only adresses long term goals; to address short term direct incentives like stronger RPS should be used and rebates for ICT improvements

\item to achieve full pottential --- full deployment has to be done in one run --- \emph{time of day pricing} and \emph{demand management} depend on implementation of smart meters --- without covering whole technology suit they are ineffective --- incentives and penalties to force innovation
\item electricity price market regulations --- dynamic pricing or pricing for carbon


\item GERMANY
\item germany set ambitious GHG abatement plans
\item they also want to phase out nuclear power generation(22\%);they emited $937 MtCO_2e$ in 2010 and 34\% of this was power generation
\item since 1990 german emision is steadly declining despite power sector seeing least improvements and is still dependant on fossil-fuels: f-f generated $347 TWh$ of electricity in 1990 and $350 TWh$ in 2010
\item GErmany introduced \emph{Energiewende}---energy transformiation --- plan to reduce GHG in energy generation by installing renewable energy sources
\item government aims at priducing 35\% of energy from renewable sources by 2010
\item nuclear energy is relatively clean and if phase out will occur and being replaced with renewable energy the overall emmision might not be significantly reduced
\item government has defined clear and ambitious roadmap to reduce emmisions, improve efficiency in energy sector.
\item in first half of 2012 share of renewable energy exceeded 25\%
\item to introduce german's plan of 80\% renewable energy in 2050 they first have to develop each sublever which plays a key role in globally working system
\item total abatement potential in germany is estimated to be $40 MtCO_2e$ in 2020

\item business case describes trial of smart grid with smart meters to deliver more detailed consumptions patterns to customers; it assumes that consumer could identify the appliance which uses the most energy --- CAN BE QUITE HARD AS YOU GET READINGS FROM THE HOME NOT PARTICULAR APPLIANCES
\item they could also avoid energy consumptions during peak hours --- HOW ? STOP USING IT?
\item it benefits generation as they can predict the usage
\item they claim that average household could save 4\% of energy with smart meter --- NO REFERENCE FOR THE SAVING --- according to BBS --- http://www.bbc.co.uk/news/business-29125809 --- THE SAVINGS ARE ONLY 2\%


\end{itemize}

\subsection{Demand management}
\subsubsection{Review}
\begin{itemize}
\item CHINA
\item iN china we can do $4 Mt CO_2e$
\item POTENTIAL OVELAY WITH [TIME-OF-DAY PRICING] AND [POWER-LOAD BALLANCING]
\item reduction of peak load electricity --- IN OTHER SUBLEVERS HAS BEEN ADDRESSED ALREADY
\item thay claim beacause of great government involvement it can be reduced by law --- not clear how

\item the USA
\item in USA potential of $21 MtCO_2e$
\item reduction of peak load electricity --- HOW TO DO IT?
\item it can be done dependant on: smart meters / smart grid infrastructure / and changing customer behaviour --- CHANGING CUSTOMER BEHAVIOUR MAY BE QUITE HARD TO ACHIEVE AND CHALLENGING NOT EASLY PREDICTABLE RESULTS --- POTENTIAL OVERLAY WITH smart grid(PPOWER GRID OPTIMIZATION) SUBLEVER

\item GERMANY
\item we can do $0.3 Mt CO_2e$
\item has small potential --- not really said how to achieve


\end{itemize}
\subsubsection{Improvement}

\subsection{Time-of-day pricing}
\subsubsection{Review}
\begin{itemize}
\item iN china we can do $79 Mt CO_2e$
\item \emph{innovative} pricing mechanisms---NOT SPECIFIED WHAT MECHANISMS--- like time of day pricing; variable electricity tarrifs constructed based on consumption time; are used to shift demand from peak to off-peak hours
\item they claim above countr-measurements could prevent supply shortage during peak hours and that \emph{total power demand} will decline --- WHY IT SHOYLD DECLINE IF THEY ONLY WANT TO SHIFT THE USAGE????
\item potential overlay with  [power-load balancing]\\ref\{...\}
\item reduce emissions because it will be more equal through day. to this end upgrade infrastructure and introduce dynamic pricing


\item the USA
\item in USA potential of $22 MtCO_2e$
\item introduce proper electricity price market --- to reduce demands on the grid and increase number of base-load --- NOT CLEAR HOW TO REDUCE THE DEMANDS IN PEAK TIMES...

\item GERMANY
\item we can do $3.2 Mt CO_2e$
\item this abatement higly depends on installation of smart meters --- which were tried to install for a long time and failed --- weak business case
\item DOES THIS ABATEMENT POTENTAIL ASSUMES SUCCESSFULL INSTALATION OF SMART METERS??????
\item installing smart meter is part of grid optimization???? hence potential overlay

\item shifting peak consumption to base-load
\item since 2011 variable tarrifs are offered

\end{itemize}
\subsubsection{Improvement}

\subsection{Power-load balancing}
\subsubsection{Review}
\begin{itemize}
\item iN china we can do $9 Mt CO_2e$
\item potential overlay with  [time of day pricing]\\ref\{...\}
\item Case study states that suppliers could quickly and reliably drive down 15-30\% the consumer's overall power consumption --- NOT CLEAR HOW, ESPECIALLLY THAT CUSTOMERS DIDNT COMPROMISE CRITICAL OPERATIONS
\item not major field for improvement in China --- they claim that it requires high energy prices to make the economics favourable --- IT KIND OF CONTRADICTS DYNAMIC ENERGY PRICING
\item raising prices might not be a good idea

\item the USA
\item in USA potential of $13 MtCO_2e$
\item they claim it has large abatement potential but is technologically limited: WHAT IS THIS ABATEMENT POTENTIAL AND WHAT TECHNOLOGIES ARE LACKING TO MAKE IT HAPPEN
\item it is highly dependant on regulatory changes --- HOW DEPENDANT??

\item GERMANY
\item we can do $0.2 Mt CO_2e$
\item ballance peak demand via off-peak storage
\item low potentail as efficient hydro and other pumped storage exsists

\end{itemize}
\subsubsection{Improvement}

\subsection{Power grid optimization}
\subsubsection{Review}
\begin{itemize}
\item CHINA
\item iN china we can do $143 Mt CO_2e$
\item minimising distribution losses by optimizing and ICT'ing grid infrastructure
\item they estimate it on 'recent efforts'...
\item improve efficiency of distribuitons and minimise losses --- how and where ICT
\item it bases on government's word that it will ICT technology to optimize grid

\item the USA
\item in USA potential of $91 MtCO_2e$
\item THEY GENERALLY SAY THAT: optimising power grid is essential for reducing transmission and distribution losses, and reducing reliability concerns --- IT SEEMS TO BE OBVIOUS FROM THE TITLE OF THE SECTION
\item in this sector ICT is important from both communication and optimisation perspective: monitoring the grid and providing real-time optimisation for its use --- BUT HOW TO OPTIMISE, AND WHAT MONIORING GIVES AS IT DOES NOT DIRECTLY REDUCE THE EMMISION
\item improve ageing infrastructure
\item in the study case: they can install all the meters and sensors but to make actual improvements you need to analyse big data, which they may not be capable of --- HOW ABOUT RENTING A CLOUD??
\item in the ``Echelon'' study case: modular processing over big data centre approach; Echelon devices smartly filter and preprocess data before sending them for analysis chence cut the amount for analysis

\item GERMANY
\item we can do $5.71 Mt CO_2e$
\item relatively efficient and modern power grid --- but losses of transmission and distribution cannot be avoided
\item smarter 2020 assumes 6\% power loss --- WHERE FROM ?????? --- out of blue --- reference 156 - "The energy and Resource Institute" --- http://www.teriin.org/upfiles/pub/papers/ft33.pdf gives apparently 4\%
\item so abatement potential potentially smaller

\end{itemize}
\subsubsection{Improvement}

\subsection{Integration of renewables in power generation}
\subsubsection{Review}
\begin{itemize}
\item CHINA
\item iN china we can do $153 Mt CO_2e$
\item this can be cut by investing in renevable sources of energy and rapid expansion
\item increasing the use of carbon free energy --- DESPITE BEING MAJOR FACTOR IN CHINA IT HAS SMALL EXPLANATION AND GENERAL NAMES DROPPED ONLY
\item it has strong government support but technical bariers like introducing ICT enabled smart-grid has to be introduced first --- POTENTIAL OVERLAY WITH [power grid optimization]

\item the USA
\item in USA potential of $197 MtCO_2e$
\item lion's share --- nearly half of the abatement potential in energy --- I would say more than half 56%
\item most of renevables are intermittent --- generation fluctuates based on weather or time of day --- its an issue as the supply must meet the load perfectly --- ICT can play important role into integrating these --- BUT HOW??????
\item argumant that ``ageing and efficient''(possibly error \[in\]efficient) grid makes this balancing more dificult --- POSSIBLE OVERLAY WITH LOAD BALANCING
\item ICT can improve integration by improving communication between grid operator and renewable power plant, analyzing weather for predicting future generation, and complex data analysis and optimisation --- BUT HOW??? --- WHAT OPTIMISATION --- COMPLEX ANALYSIS OF WHAT --- THE PRODUCTION CAN BE PREDICTED BASED ON WEATHER CONDITIONS SO WHAT DOES COMMUNICATION GIVE TO US???
\item replace grid electricity with carbon-free electricity --- depends on RPS at state level
\item adapting grid to intermittent generation and new transmission
\item customers can sell the overhead of energy to the grid but the prices are very low
\item it is also not profiteble due to large upfront cost

\item GERMANY
\item we can do $28.8 Mt CO_2e$
\item they have plan to introduce new renewable sources but simultaneously phase out nuclear energy --- ICT will be needed to integrate these renewables into the grid and maximize its efficiency
\item but possibly renewables will replace only phased-out nuclear hence have no influence of fossil-fuel production --- only little improvement
\item IT'S UNCLEAR --- GERMANY STAT THAT !THEY WILL! CHANGE NUCLEAR FOR RENEVABLA SO THIS ABATEMENT POTENTIAL IS ONLY THE DIFFERENCE BETWEEN NUCLEAR AND RENEWABLE OR THEY ASSUME THAT NUCLEAR WILL REMAIN AND IT WIL BE FOSSIL-FUEL REPLACED BY RENEWABLE

\item increasing and optimizaiton of emmision-neutral power use
\item necessary due to nuclear phase-out

\item LACK OF CHEAP STORAGE FOR ENERGY PRODUCED BY PV.
\item need for nort-south transmission lines and large-scale energy storage


\end{itemize}
\subsubsection{Improvement}

\subsection{Virtual power plant}
\subsubsection{Review}
\begin{itemize}
\item CHINA
\item iN china we can do $2 Mt CO_2e$
\item requires building advanced-grid-infrastructure which are not YET available --- few details????
\item ITS HARD TO IMPLEMENT DUE TO MONOPOLY --- NOT CLEAR HOW SMALL LOCAL GRID OF RENEVABLES CAN BE HARDMED BY MONOPOLY

\item the USA
\item in USA potential of $6 MtCO_2e$
\item once again dependant on government regulations and advanced smart grid technology --- BUT HOW, WHAT REGULATIONS, WHAT TECHNOLOGY

\item GERMANY
\item we can do $1.7 Mt CO_2e$

\item it requires advanced grid infrastructure which germany is lacking
\item limited impact on direct emission reduction

\end{itemize}
\subsubsection{Improvement}

\subsection{Integration of off-grid renewables and storage}
\subsubsection{Review}
\begin{itemize}
\item iN china we can do $1 Mt CO_2e$; its called here \emph{island grid} --- inconsistency
\item they claim it has small potential due to number of diesel-generators and off-grid applications. DOES IT MEAN THAT THERE IS SMALL NUMBER OF OFF-GRID GENERATORS HENCE NO ROOM FOR IMPROVEMENT OR IT IS NOT POSSIBLE TO REPLACE THIS GENERATORS --- NOT CLEAR
\item THE NAME SUGGEST THEY ARE MAINLY ON ISLAND --- IS IT DIFFICULT TO GIVE THEM A CABLE RO WHAT

\item the USA
\item in USA potential of --- NOT RELEVANT!!!
\item this sublever is not important in US due to widely available grid-connected electricity

\item GERMANY
\item we can do $0 Mt CO_2e$ --- irrelevant
\item reliable grid available videly hence none impact

\end{itemize}
\subsubsection{Improvement}

\section{Conclusions}
Based on the assessment of the abatement potential we then require from you an interpretation of the report for our department in which you summarise your analysis of the strengths and weaknesses of the SMARTer2020 report and identify possible conclusions and actions for policy making. Among other things our policy analysts are interested in areas where financial incentives to the uptake of innovations are lacking and where government intervention might be required.\\

They can collect the data but without datacentres and tools they won't be able to make any sense of them and use them to improve the grid

\section{Tool}
Ipython julia with markup+latex\\
Make sure, I can follow where assumptions came from

\section{Summary}

\end{document}
